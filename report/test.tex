\documentclass[conference]{IEEEtran}

% --- Packages you might need ---
\usepackage[utf8]{inputenc}   % Encoding
\usepackage{amsmath,amssymb}  % Math symbols
\usepackage{graphicx}         % Figures
\usepackage{cite}             % Better citations
\usepackage{hyperref}         % Clickable links

% --- Title and author ---
\title{Exploration of DyNA PPO with Dynamic Ensemble}

\author{
    \IEEEauthorblockN{Leihao (Eric) Lin\IEEEauthorrefmark{1}}
    \IEEEauthorblockA{\IEEEauthorrefmark{1}Department of Computer Science, Western University \\
    Email: llin286@uwo.ca}
}

\begin{document}

\maketitle

\begin{abstract}
This is the abstract of the paper. It should briefly summarize the problem, method, and results.
\end{abstract}

\section{Introduction}
This document is an example of writing in IEEE two-column format.  
You can add references \cite{IEEEexample} and equations:
\[
    E = mc^2
\]

\section{Backgrounds and Related Works}


\section{Goal and Objectives}
Figures can be inserted as follows (Figure~\ref{fig:example}):

\section{Methodology}

\section{Results}


\section{Conclusion}
Your conclusion goes here.

\bibliographystyle{IEEEtran}
\bibliography{references} % You can create a references.bib file

\end{document}