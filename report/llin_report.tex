\documentclass[conference]{IEEEtran}

% --- Packages you might need ---
\usepackage[utf8]{inputenc}   % Encoding
\usepackage{amsmath,amssymb}  % Math symbols
\usepackage{graphicx}         % Figures
\usepackage{cite}             % Better citations
\usepackage{hyperref}         % Clickable links

% --- Title and author ---
\title{Exploration of DyNA PPO with Dynamic Ensemble}

\author{
    \IEEEauthorblockN{Leihao (Eric) Lin\IEEEauthorrefmark{1}}
    \IEEEauthorblockA{\IEEEauthorrefmark{1}Department of Computer Science, Western University \\
    Email: llin286@uwo.ca}
}

\begin{document}

\maketitle

% Abstract
\begin{abstract}
This is the abstract of the paper. 
\end{abstract}


% introduction
\section{Introduction}
Introduction of the project
...



%Background and Related Works
\section{Backgrounds and Related Works}
Using PPO, a stable policy-gradient RL method to solve the black box optimization of biological sequence design. 
It proposes DyNA PPO, a variant of Proximal Policy Optimization:
\begin{itemize}
	\item Learns a surrogate reward model through supervised regression on data collected.
	\item Using cross validated R2, selects from a pool of candidate regressors whose predictions are above a threshold, and uses their ensemble average as a simulator for policy updates.
	\item Falls back to model-free PPO when no accurate surrogate is available, avoiding model bias.
	\item Adds exploration bonus penalizing proposals too similar to past sequences to encourage diversity.
\end{itemize}


% Goal and Objectives
\section{Goal and Objectives}
\begin{enumerate}
    \item Reproduce the standard DyNA PPO from the original paper.
    
    \item Formulate the surrogate ensemble reward $r'(x)$ with weights $w_i$ chosen to minimize a combination of surrogate bias and variance under cross-validation estimates. 
    
    \item Combine several surrogate models into one reward function:
    \[
        r'(x) = \sum_{i=1}^K w_i f'_i(x).
    \]
    
    \item Prove a bound on the regret of the model-based policy update step that decomposes into:
    \begin{enumerate}
        \item model bias terms, and 
        \item policy-optimization error,
    \end{enumerate}
    showing conditions under which weighted ensembling strictly improves sample efficiency over uniform averaging.
    
    \item Define regret as the loss in reward by following the approximate surrogate-based policy update, compared to using the true fitness function at every step. 
    
    \item Decompose regret into:
    \begin{itemize}
        \item How wrong the surrogate is, and 
        \item How imperfect our policy update on that surrogate is.
    \end{itemize}
    
    This allows us to optimally choose model weights to shrink model bias, rather than equally averaging all models. As a result, the overall regret is smaller and fewer real samples are needed to learn an effective policy.
    
    \item Implement the weighted DyNA PPO algorithm, integrating it into the existing PPO plus surrogate loop. Re-estimate weights each round automatically via a small convex optimization step.
    
    \item Add an extra optimization step each round to resolve for the best ensemble weights.
    
    \item Empirically compare the weighted DyNA PPO against the standard DyNA PPO on benchmark tasks.
\end{enumerate}

\section{Methodology}




% Experiments and Results
\section{Experiments and Results}

\subsection{Base Comparison}
We compare the following approaches:
\begin{itemize}
    \item Standard with average ensemble
    \item R\textsuperscript{2}-based weighted ensemble
    \item Dynamic optimal ensemble
    \item Pure PPO
\end{itemize}

Metrics to track:
\begin{itemize}
    \item Final best reward achieved
    \item Convergence speed (rounds to reach 90\% of best)
    \item Reward Variance
\end{itemize}

\subsection{Ablation Study}
We test the importance of each component by removing them individually.  
Configurations:
\begin{itemize}
    \item no\_warmup
    \item no\_diversity\_penalty
    \item fixed\_threshold
    \item uniform\_weights\_only
    \item no\_context\_encoding
\end{itemize}



\subsection{Model Contribution Analysis}
We analyze the contribution of each model over time.  
For each round, we log:
\begin{itemize}
    \item Individual model R\textsuperscript{2} scores
    \item Assigned weights
    \item Prediction accuracy
\end{itemize}

Visualization: stacked area charts will show the weight distribution over time.






% Conclusion
\section{Conclusion}
Your conclusion goes here.

\bibliographystyle{IEEEtran}
\bibliography{references} % You can create a references.bib file

\end{document}